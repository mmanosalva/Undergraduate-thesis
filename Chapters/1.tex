%!TEX root = ../main.tex

\thispagestyle{empty}
\vspace{-0.5cm}

\cleanchapterquote{La matemática posee no solo verdad, sino también belleza suprema; una belleza fría y austera, como aquella de la escultura, sin apelación a ninguna parte de nuestra naturaleza débil, sin los adornos magníficos de la pintura o la música, pero sublime y pura, y capaz de una perfección severa como solo las mejores artes pueden presentar}{Bertrand Russel}{}\\

La teoría de  números fue llamada por Gauss, la reina de las matemáticas, quizá por la simplicidad de su objeto o la elegancia y diversidad de sus métodos, que la convierten en una de las áreas más fascinantes del universo matemático. Sin embargo no solo podemos resaltar la gran variedad de objetos matemáticos que intervienen en ella, sino la capacidad que tiene para conectarlos, que el teorema de Dirichlet junta de manera sorprendente ideas del álgebra, con el análisis de Fourier, que  en un principio aparece para el estudio de las ecuaciones diferenciales parciales, que el teorema de los números primos resulte ser una consecuencia de la no nulidad de una función en una recta del plano complejo, que los ceros de  una función controlen el comportamiento de los números primos... Estas entre muchas otras, forman parte de las fascinantes y inesperadas conexiones que aparecen al adentrarnos en las ideas un poco más modernas de la teoría de números.\\

Es sorprendente que el análisis y sus objetos sean capaces de decir algo sobre la naturaleza, en un principio discreta, de los números naturales, quizás la primera relación que me cautivó fue el producto de Euler, ver una función de variable compleja ser escrita como un producto sobre los números primos es algo maravilloso, el cómo eran posibles estas conexiones fue lo que siempre quise comprender, y que de cierto modo al adentrarme en la teoría analítica de números, pude  lograr, pero para entender cómo nace esta relación debemos estudiar algunas de las ideas de Euler. \\

Alrededor del año 300 a.c Euclides prueba que hay infinitos números primos, establece que si los primos son finitos, entonces el producto $p_1\ldots p_n+1$ no es divisible por ningún primo $p_1,\ldots,p_n$, de esta manera siempre se puede construir un número primo adicional. En el siglo XVIII Euler prueba que hay infinitos primos usando la divergencia de la serie armónica, si asumimos que hay un número finito de números primos, entonces el siguiente producto es finito:

$$\prod_p \dfrac{1}{1-p^{-1}}$$

Ahora note que el término del producto es a lo que converge una serie geométrica y dado que $|p^{-1}|<1$, entonces:

\begin{align*}
    \infty>\prod_p \dfrac{1}{1-p^{-1}}&=\prod_p \left(\sum_{k=0}^{\infty}\dfrac{1}{p^k}\right)=\prod_p \left(1+\dfrac{1}{p}+\dfrac{1}{p^2}+\dfrac{1}{p^3}+\ldots\right)\\
    &=\sum_{n=1}^{\infty}\dfrac{1}{n}\\
    &=\infty
.\end{align*}

Ya que todo número natural puede escribirse de manera única como producto de potencias de primos, esto nos lleva a una evidente contradicción. Euler consigue este argumento ya que venía de estudiar problemas similares, como la convergencia de la serie:

$$\sum_{n=1}^{\infty}\frac{1}{n^2}=\frac{\pi^2}{6}$$

La idea que tuvo Euler provenía de estudiar la serie de Taylor de la $\sin x$:

$$\sin x=x-\frac{x^3}{3!}+\frac{x^5}{5!}-\frac{x^7}{7!}\pm \ldots$$

Así:

$$\frac{\sin x}{x}=1-\frac{x^2}{3!}+\frac{x^4}{5!}-\frac{x^6}{7!}\pm\ldots$$

En este punto es donde hace un salto de fe pensando que el polinomio de Taylor se puede escribir como un producto infinito si lo factorizamos sobre sus raíces, ie. Las  raíces de $\dfrac{\sin x}{x}$, asume que lo que ocurre para polinomios finitos también se tiene para infinitos, obteniendo que:

\begin{align*}
    \frac{\sin x}{x}&=1-\frac{x^2}{3!}+\frac{x^4}{5!}-\frac{x^6}{7!}\pm\ldots\\
    &=\left(1+\frac{x}{\pi}\right)\left(1-\frac{x}{\pi}\right)\left(1+\frac{x}{2\pi}\right)\left(1-\frac{x}{2\pi}\right)\left(1+\frac{x}{3\pi}\right)\left(1-\frac{x}{3\pi}\right)\ldots\\
&=\left(1-\frac{x^2}{\pi^2}\right)\left(1-\frac{x^2}{2^2\pi^2}\right)\left(1-\frac{x^2}{3^2\pi^2}\right)\ldots
\end{align*}

Luego comparando el coeficiente de $x^2$ en la serie con el de el producto:

$$\frac{1}{3!}=\frac{1}{\pi^2}\left(1+\frac{1}{2^2}+\frac{1}{3^2}+\ldots\right)$$

Esta idea que le daría la ``solución'' al problema se formaliza a través del teorema de factorización de Weierstrass. Euler seguiría estudiando este problema por mucho tiempo y lo generalizaría a través de la serie absolutamente convergente:

$$\zeta(s)=\sum_{n=1}^{\infty}\frac{1}{n^s}, \quad s>1$$

Tiempo después encuentra una fórmula para obtener los valores de esta función en los números pares, ie. $\zeta(2s)$ y también obtuvo su desarrollo como producto:

$$\zeta(s)=\prod_{p}\frac{1}{1-p^{-s}}$$

Esto le permitió demostrar la divergencia de la serie $\displaystyle \sum_p\frac{1}{p}$, un argumento directo y totalmente analítico de que hay infinitos números primos.\\

Estas ideas llamaron la atención de dos matemáticos muy importantes, Dirichlet y Riemann. Dirichlet usó estas ideas para probar su teorema de progresiones aritmética, Riemann por otro lado estudió íntimamente la función $\zeta(s)$, le asignó a $s$ un número complejo y también la llevó a tener su fama actual al lanzar su conocida conjetura, pero, ¿esto qué tiene que ver con el teorema de los números primos?.\\

Conjeturado de manera independiente por Gauss (1792) y Legendre (1798), el teorema de los números primos nos permite entender el comportamiento asintótico de la función contadora de  primos $\pi(x)$, nos dice  que para números grandes, la cantidad de primos menores que $x$ se puede aproximar por $\dfrac{x}{\log x}$, escrito de manera formal:

$$\lim_{x \to \infty}\dfrac{\pi(x)\log x}{x}=1 \quad \text{o en notación asintótica} \quad \pi(x)\thicksim \dfrac{x}{\log x}$$

\vspace*{1cm}

Una interpretación heurística de este teorema viene de estudiar la densidad de un conjunto de números naturales. Dado $N\subseteq \N$, la \text{densidad natural} de $N$ la definimos como:

$$d=\lim_{n\to \infty}\frac{|\{m\leq n \text{ | }m\in N\}|}{n}, \quad \text{siempre que exista el límite}$$

Estudiar la probabilidad de, por ejemplo, que un entero sea divisible por un primo $p$ será equivalente a calcular la densidad del conjunto de enteros que cumple esta propiedad, veamos esto. Dado $n$, sea  $c$ el número de enteros $m\leq n$ tal que $p\mid m$, sabemos por un simple conteo que:

$$\frac{n}{p}-1\leq c\leq\frac{n}{p}+1$$

Luego, $d=\displaystyle\lim_{n \to \infty}\dfrac{c}{n}=\dfrac{1}{p}$ por el criterio de comparación. Esto nos dice que la probabilidad de que un entero no sea divisible por $p$ es $1-\dfrac{1}{p}$, sabemos además que este evento es excluyente, así... La probabilidad de que un número sea primo viene dada por:

$$\displaystyle\prod_{p<n}\left( 1-\frac{1}{p} \right)$$

Queremos ver que esto crece como $\dfrac{1}{\log n}$, para esto podemos invertir el producto anterior, nuevamente tenemos que:

$$\prod_{p<n} \dfrac{1}{1-p^{-1}}=\prod_{p<n} \left(1+\dfrac{1}{p}+\dfrac{1}{p^2}+\dfrac{1}{p^3}+\ldots\right)=\sum_{k<n}\frac{1}{k}=H(n)$$

En el siguiente capítulo veremos justamente que $H(n)\thicksim \log n$, esto concluye lo que queríamos ver: \text{la probabilidad de que un número sea primo es} $\dfrac{1}{\log n}$, \textit{es inversamente proporcional a su longitud}. ie:

 $$\pi(x)\thicksim \dfrac{x}{\log x}$$

Pero, ¿cómo se puede demostrar algo así?, el camino a seguir en un principio es sorprendente y viene del estudio de la función de $\zeta(s)$, vista como función de variable compleja absolutamente convergente si $\Re(s)>1$. El primero en mostrar que estudiar esta función daba un camino hacia una prueba del teorema de los números primos fue Riemann en su famoso articulo "Sobre la cantidad de primos menores que una magnitud dada" \cite{riemann1990ueber}. Allí Riemann presentaría muchas ideas, pero no las desarrollaría y fue el trabajo de los matemáticos en los siguientes 50 años llegar a una demostración, trabajo que culminaría en las demostraciones Hadammar y de la Vallée Poussin que aparecen en 1896, la prueba, vendría del hecho de que $\zeta(1+it)\neq 0$, es decir, la función $\zeta$ no se anulaba en la recta vertical de los complejos con parte real 1, sobre el plano complejo, algo sencillamente maravilloso. Veremos al final de este trabajo la forma en que este teorema se extiende a progresiones aritmética $a+kq$ con $(a,q)=1$:

$$\pi_{a,q}(x)\thicksim \frac{x}{\varphi(q)\log x}$$

Donde $\varphi$ es la función phi de Euler, y como hay $\varphi(q)$ clases generadoras de primos, entonces los primos se distribuyen uniformemente en las clases módulo q.\\

En el capítulo 1 presentaremos algunos preliminares que se pueden consultar en el contenido y estudiaremos un poco la función $\zeta(s)$ y su derivada logarítmica $\dfrac{\zeta^{\prime}(s)}{\zeta(s)}$, veremos que el TNP es equivalente a la afirmación $\psi(x)\thicksim x$, función que también estudiaremos allí. El capítulo 2 será para presentar una prueba del teorema de Dirichlet, las ideas subyacentes y los preliminares de la  prueba también se desarrollarán allí, en los capítulos 3 y 4 se desarrollarán las pruebas del TNP y el TNP sobre progresiones aritmética, estudiaremos la teoría Tauberiana, que nos permitirá dar una prueba sencilla del TNP y donde casi toda  la variable compleja estará escondida en el teorema de Wiener-Ikehara que también presentaremos allí junto con algunas aplicaciones.