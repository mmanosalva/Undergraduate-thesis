%!TEX root = ../main.tex

\thispagestyle{empty}
\vspace{-0.5cm}

\cleanchapterquote{La matemática posee no solo verdad, sino también belleza suprema; una belleza fría y austera, como aquella de la escultura, sin apelación a ninguna parte de nuestra naturaleza débil, sin los adornos magníficos de la pintura o la música, pero sublime y pura, y capaz de una perfección severa como solo las mejores artes pueden presentar}{B. Russel}{}

La teoría de  números fue llamada por Gauss, la reina de las matemáticas, quizá por la simplicidad de su objeto o la elegancia y diversidad de sus métodos, que la convierten en una de las áreas más fascinantes del universo matemático. Sin embargo no solo podemos resaltar la gran variedad de objetos matemáticos que intervienen en ella, sino la capacidad que tiene para conectarlos, por ejemplo, el teorema de Dirichlet junta de manera sorprendente ideas del álgebra, con el análisis de Fourier, que  en un principio aparece para el estudio de las ecuaciones diferenciales parciales, el que el teorema de los números primos resulte ser una consecuencia de la no nulidad de una función en una recta del plano complejo, que los ceros de  una función controlen el comportamiento de los números primos... Estas entre muchas otras, forman parte de las fascinantes y inesperadas conexiones que aparecen al adentrarnos en las ideas un poco más modernas de la teoría de números.\\

Es sorprendente que el análisis y sus objetos sean capaces de decir algo sobre la naturaleza, en un principio discreta, de los números naturales, quizás la primera relación que me cautivó fue el producto de Euler, ver una función de variable compleja ser escrita como un producto sobre los números primos es algo maravilloso, el cómo eran posibles estas conexiones fue lo que siempre quise comprender, y que de cierto modo al adentrarme en la teoría analítica de números, pude  lograr, pero para entender cómo nace esta relación debemos estudiar algunas de las ideas de Euler. \\

Alrededor del año 300 a.c Euclides prueba que hay infinitos números primos, establece que si los primos son finitos, entonces el producto $p_1\ldots p_n+1$ no es divisible por ningún primo $p_1,\ldots,p_n$, de esta manera siempre se puede construir un número primo adicional. En el siglo XVIII Euler prueba que hay infinitos primos usando la divergencia de la serie armónica, si asumimos que hay un número finito de números primos, entonces el siguiente producto es finito:

$$\prod_p \dfrac{1}{1-p^{-1}}$$

Ahora note que el término del producto es a lo que converge una serie geométrica y dado que $|p^{-1}|<1$, entonces:

\begin{align*}
    \infty>\prod_p \dfrac{1}{1-p^{-1}}&=\prod_p \left(\sum_{k=0}^{\infty}\dfrac{1}{p^k}\right)=\prod_p \left(1+\dfrac{1}{p}+\dfrac{1}{p^2}+\dfrac{1}{p^3}+\ldots\right)\\
    &=\sum_{n=1}^{\infty}\dfrac{1}{n}\\
    &=\infty
.\end{align*}

Ya que todo número natural puede escribirse de manera única como producto de potencias de primos, esto nos lleva a una evidente contradicción. Euler consigue este argumento ya que venía de estudiar problemas similares, como la convergencia de la serie:

$$\sum_{n=1}^{\infty}\frac{1}{n^2}=\frac{\pi^2}{6}$$

La idea detrás de este problema viene de estudiar la serie de Taylor de $\sin x$:

$$\sin x=x-\frac{x^3}{3!}+\frac{x^5}{5!}-\frac{x^7}{7!}\pm \ldots$$

Así:

$$\frac{\sin x}{x}=1-\frac{x^2}{3!}+\frac{x^4}{5!}-\frac{x^6}{7!}\pm\ldots$$

Euler hace un salto de fe pensando que el polinomio de Taylor se puede escribir como un producto infinito si lo factorizamos sobre sus raíces, ie. Las  raíces de $\dfrac{\sin x}{x}$, asume que lo que ocurre para polinomios finitos también se tiene para infinitos...

\begin{align*}
    \frac{\sin x}{x}&=1-\frac{x^2}{3!}+\frac{x^4}{5!}-\frac{x^6}{7!}\pm\ldots\\
    &=\left(1+\frac{x}{\pi}\right)\left(1-\frac{x}{\pi}\right)\left(1+\frac{x}{2\pi}\right)\left(1-\frac{x}{2\pi}\right)\left(1+\frac{x}{3\pi}\right)\left(1-\frac{x}{3\pi}\right)\ldots\\
&=\left(1-\frac{x^2}{\pi^2}\right)\left(1-\frac{x^2}{2^2\pi^2}\right)\left(1-\frac{x^2}{3^2\pi^2}\right)\ldots
\end{align*}

Luego comparando el coeficiente de $x^2$ en la serie con el de el producto:

$$\frac{1}{3!}=\frac{1}{\pi^2}\left(1+\frac{1}{2^2}+\frac{1}{3^2}+\ldots\right)$$

Lo que le daría la ``solución'' al problema y lo motivaría a generalizarlo con la serie absolutamente convergente:

$$\zeta(s)=\sum_{n=1}^{\infty}\frac{1}{n^s}, \quad s>1$$

Euler encuentra una fórmula para obtener los valores de esta función en los pares, ie. $\zeta(2s)$ y también obtuvo su desarrollo como producto:

$$\zeta(s)=\prod_{p}\frac{1}{1-p^{-s}}$$

Esto le permitió demostrar la divergencia de la serie $\displaystyle \sum_p\frac{1}{p}$, un argumento directo y totalmente analítico de que hay infinitos números primos.\\

Estas ideas llamaron la atención de Dirichlet y Riemann, siendo este último quien estudió íntimamente la función $\zeta$ y la llevó a la fama que posee actualmente, pero, ¿esto qué tiene que ver con el teorema de los números primos?.\\

Conjeturado de manera independiente por Gauss (1792) y Legendre (1798), el teorema de los números primos nos permite entender el comportamiento asintótico de la función contadora de  primos $\pi(x)$, nos dice  que para números grandes, la cantidad de primos menores que $x$ se puede aproximar por $\dfrac{x}{\log x}$, escrito de manera formal:

$$\lim_{x \to \infty}\dfrac{\pi(x)\log x}{x}=1 \quad \text{o en notación asintótica} \quad \pi(x)\thicksim \dfrac{x}{\log x}$$\\

Pero, ¿cómo se puede demostrar algo así?, el camino a seguir en un principio es sorprendente y viene del estudio de la función de $\zeta(s)$, vista como función de variable compleja absolutamente convergente si $\Re(s)>1$. El primero en mostrar que estudiar esta función daba un camino hacia una prueba del teorema de los números primos fue Riemann en su famoso articulo "Sobre la cantidad de primos menores que una magnitud dada" \cite{riemann1990ueber}. Allí Riemann presentaría muchas ideas, pero no las desarrollaría y fue el trabajo de los matemáticos en los siguientes 50 años llegar a una demostración, trabajo que culminaría en las demostraciones Hadammar y de la Vallée Poussin que aparecen en 1896, la prueba, vendría del hecho de que $\zeta(1+it)\neq 0$, es decir, la función $\zeta$ no se anulaba en la recta vertical de los complejos con parte real 1, sobre el plano complejo, algo sencillamente maravilloso.\\

Dirichlet por otro lado, en 1837 había probado que dados $a,q \in \N$ tales que $(a,q)=1$, entonces hay infinitos primos en la progresión $a+kq$. La prueba sería basada en las ideas de Euler, estudiar  la divergencia de:

$$\sum_{p \equiv a \bmod q}\frac{1}{p}$$

Sin embargo, Dirichlet, en el artículo de su prueba del teorema de progresiones aritmética afirma que en un principio la prueba que presentó no era la que originalmente pensó; comenta que la prueba usaba argumentos un poco más indirectos y dependía del hecho de que la función $L(s,\chi)$, no se anulaba en los complejos con $\Re(s)=1$, por lo que finalmente, al no poder probarlo, optó por un argumento distinto.\\

Algo natural que se preguntaron los matemáticos es cómo se distribuyen los primos de dicho conjunto o si aquí es válido el TNP, de la Vallée Poussin también en 1896 demostraría que la función contadora de primos restringida sobre la progresión aritmética, ie. $\pi_{a,q}(x)$ tiene el comportamiento asintótico:\\

$$\pi_{a,q}(x)\thicksim \frac{x}{\varphi(q)\log x} $$

Donde $\varphi$ es la función phi de Euler, y como hay $\varphi(q)$ clases generadoras de primos, entonces los primos se distribuyen uniformemente en las clases módulo q.\\

La prueba de este resultado no es distinta de la del teorema de los números primos ya que en esencia depende del hecho de que $L(s,\chi)\neq 0$ si $\Re(s)=1$. Así, nuestro trabajo será construir las herramientas para esclarecer estas ideas que concluirán cuando probemos la no nulidad de $L(\chi,s)$ y $\zeta(s)$ en $\Re(s)=1$, la idea es que al final de este trabajo para el lector no sea confuso que el TNP se siga de esto, para ello estudiar las ideas subyacentes será muy importante. Resultará conveniente que el lector esté familiarizado con algunos conceptos de variable compleja, análisis real y teoría de grupos, ya que aunque presentaremos gran parte de los preliminares aquí, no profundizaremos en ellos como sí se haría en un curso de análisis o álgebra.\\

En el capítulo 1 presentaremos algunos preliminares que se pueden consultar en el contenido y estudiaremos un poco la función $\zeta(s)$ y su derivada logarítmica $\dfrac{\zeta^{\prime}(s)}{\zeta(s)}$, veremos que el TNP es equivalente a la afirmación $\psi(x)\thicksim x$, función que también estudiaremos allí. El capítulo 2 será para presentar una prueba del teorema de Dirichlet, las ideas subyacentes y los preliminares de la  prueba también se desarrollarán allí, en los capítulos 3 y 4 se desarrollarán las pruebas del TNP y el TNP sobre progresiones aritmética, estudiaremos la teoría Tauberiana, que nos permitirá dar una prueba sencilla del TNP y donde casi toda  la variable compleja estará escondida en el teorema de Wiener-Ikehara que también presentaremos allí junto con algunas aplicaciones.